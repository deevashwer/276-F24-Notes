\let\counterwithout\relax
\let\counterwithin\relax

\let\proof\relax
\let\endproof\relax
\let\example\relax
\let\endexample\relax

%% Tikz configuration
\usetikzlibrary{shapes.geometric, arrows}

% Change thanks markers to numbers
\makeatletter
\let\@fnsymbol\@arabic
\makeatother

%% LLNCS hyperref requirement
\renewcommand\UrlFont{\color{blue}\rmfamily}

%% Eurocrypt requirement
\pagestyle{plain}


%% Allow align environment to span multiple pages
\allowdisplaybreaks

%% Theorem Environments

% Bold optional theorem titles
\makeatletter
\def\th@definition{%
  \thm@notefont{}% same as heading font
  \normalfont % body font
}
\makeatother

% No italics
% \theoremstyle{definition}

% Rename environments
\newtheorem{assumption}{Assumption}
% \newtheorem{construction}{Construction}
\newtheorem{attention}{Remark}
\newtheorem{notation}{Notation}

% Algorithm counter to construction
\makeatletter 
\renewcommand\thealgorithm{\theconstruction.\arabic{algorithm}} 
\@addtoreset{algorithm}{construction}
\makeatother

%% Annotations
\newcommand\todo[1]{\textcolor{red}{TODO: #1}}
\definecolor{comment-color}{RGB}{68, 59, 141}
\renewcommand{\algorithmiccomment}[1]{\textcolor{comment-color}{// #1}}
\newcommand\blue[1]{\textcolor{blue}{#1}}
\newcommand{\mathhl}[1]{\colorbox{yellow}{$\displaystyle #1$}}



%% Diagram Macros

\newcommand{\rightarr}[1]{\sendmessageright{length=2.5cm, top=\colorbox{white}{$#1$}, topstyle={yshift=-9}}}
\newcommand{\leftarr}[1]{\sendmessageleft{length=2.5cm, top=\colorbox{white}{$#1$}, topstyle={yshift=-9}}}

\newcommand{\cfbox}[2]{
    \colorlet{currentcolor}{.}
    {\color{#1}
    \fbox{\color{currentcolor}#2}}
}

\newcommand{\En}{\mathcal{K}}
\newcommand{\Gn}{\mathcal{G}}
\newcommand{\Po}{\mathcal{P}}
\newcommand{\V}{\mathcal{V}}


\newcommand{\niksdef}[6]{
For all expected polynomial-time adversaries 
$\mathcal{P}^*$ 
there exists an expected polynomial-time extractor
$\mathcal{E}$ such that
    \[
    \Pr_{\mathsf{r}}
    \left[
      \begin{array}{l}
        #6
      \end{array}
      \middle\vert
      \begin{array}{l}
        % Generator
        \mathsf{pp} \gets \mathcal{G}(1^{\lambda}, N),\\ 
        % Adversarial Statement, Proof
        (#1, #2, #4) \gets \mathcal{P}^*(\mathsf{pp}, \mathsf{r}),\\
        % Key Generator
        (\pk, \vk) \gets \En(\pp, #1),\\
        % Precondition
        #5,\\
        % Extractor
        #3 \gets \mathcal{E}(\pp, \mathsf{r})
      \end{array}
    \right]
    \approx 
    1
    \]
    where $\mathsf{r}$ denotes an arbitrarily long random tape.
}


% \newcommand{\pp}{\mathsf{pp}}
% \newcommand{\pk}{\mathsf{pk}}
% \newcommand{\vk}{\mathsf{vk}}

% \newcommand{\R}{\mathcal{R}}

\newcommand{\FP}{F'}
\newcommand{\io}{\mathsf{x}}
\newcommand{\fu}{\mathsf{u}}
\newcommand{\fw}{\mathsf{w}}
\newcommand{\acc}{\mathsf{U}}
\newcommand{\aw}{\mathsf{W}}
\newcommand{\trivi}{\fu_{\bot}}
\newcommand{\trivw}{\fw_{\bot}}
\newcommand{\fold}{\mathsf{NIFS}}
\newcommand{\snark}{\mathsf{SNARK}}
\newcommand{\RIVC}{\R_{\mathsf{IVC}}}
\newcommand{\Str}{\mathsf{s}}
\newcommand{\com}{\mathsf{com}}

% \newcommand{\Gen}{\mathsf{Gen}}
\newcommand{\Commit}{\mathsf{Commit}}

% \newcommand{\negl}[1]{\mathsf{negl}(#1)}

\section{Recursive Proofs of Knowledge}

In this lecture, 
we discuss recursive zero-knowledge succinct non-interactive
arguments of knowledge (zkSNARKs).
%
We assume familiarity with zkSNARKs.

\subsection{Introduction}

% Zero-Knowledge Proofs
Succinct non-interactive arguments of knowledge are short certificates that attest to the correct execution of a computation without revealing any secret inputs. 
%
Today, 
zero-knowledge proofs are being used 
to secure \emph{billions} of dollars worth of assets~\cite{zerocash, stark}.
%
Zero-knowledge proofs 
enable a new class of secure applications 
with enhanced integrity and privacy guarantees
such as verifiable databases~\cite{zkvsql, vsql, accountablestorage, integridb},
private voting protocols~\cite{privatevoting},
anonymous credentials~\cite{cinderella, dacreds},
and
private cryptocurrencies~\cite{zerocash, pinocchiocoin, stark}.

% Proving a Function
More technically, 
SNARKs 
(for circuit-satisfiability)
allow a prover to demonstrate that it knows a secret witness $w$
such that for prescribed circuit $F$ and prescribed input and output pair $(x, y)$ that $F(x, w) = y$.
%
% Proving Recursion
However, today, we will be interested in
proving \emph{recursive} computation. 
Without loss of generality, we are interested in proving \emph{tail}
recursion,
that is, we want to prove
(unbounded) recursive applications 
of a function $F$.
Unbounded recursion
in general allows us to implement more complex programming patterns such as 
$\mathsf{for}$ and $\mathsf{while}$, 
which are not bounded ahead of time.
This allows us to prove stateful computations with dynamic control flow.
%% Use cases
In practice, 
proving recursion allows us to recursively prove blockchain updates, verifiable delay functions, and even a universal machine,
where each recursive step is a single cycle of a CPU.


% Naive Solution
Historically, 
the best known approach to design a proof system for recursive applications of a function $F$ 
was to unroll the entire execution $F \circ F \circ \cdots \circ F$ into a monolithic arithmetic circuit,
and then use a standard proof system with succinct proofs for circuit satisfiability.
%
Unfortunately,
this would necessarily mean that the prover's space complexity would scale with the \emph{entire trace} of the computation.
%
Moreover, 
in the setting of preprocessed arguments of knowledge
(where the prover and verifier would process the circuit into a succinct key to use for multiple inputs)
this fixed the recursion-depth ahead of time,
which often does not reflect practice.

% Recursive Solution Overview
The first breakthrough was due to Valiant~\cite{valiant} in 2012, 
who proposed incrementally verifiable computation (IVC),
which reflected the recursive structure of the computation
into the proof itself:
%
Given a succinct proof $\pi_{i}$ attesting to $i$ steps of computation,
the prover can write a succinct proof $\pi_{i + 1}$ that attests to $i + 1$ steps
by proving the correct execution of an arithmetic circuit 
that runs the latest step of computation,
and checks $\pi_i$ 
(using the proof system's verifier).
This avoids having to fix the recursion depth ahead of time,
while ensuring that the prover's space complexity only scales with a single step of execution.

Although undoubtedly elegant, 
Valiant's technique introduces a subtle issue:
Proofs of knowledge must satisfy a stronger notion of soundness 
known as \emph{knowledge-soundness}.
%
Essentially,
a proof system is considered knowledge-sound 
if, 
for any successful prover with some secret input to the computation, 
there exists a corresponding \emph{extractor} that,
with at most polynomial overhead,
can retrieve this secret input given access to the ``source code'' of the prover.
%
This extractor-based definition becomes problematic in the recursive setting:
Recursive proofs require \emph{recursive extraction} 
in which the extractor for step $i$ 
plays the successful prover for the extractor at step $i - 1$. 
This incurs a polynomial blowup in the extractor for each successive recursive step. 
In particular, 
this results in a final extractor that runs in exponential-time with respect to the recursion-depth, 
which disqualifies it as a valid extractor. 
%
To account for this issue, 
Valiant's original technique 
(and modern techniques) 
can only provably guarantee logarithmic-depth recursion in standard models.

\subsection{Preliminaries}

We operate in the \textit{preprocessing model}, which means that a trusted party will be responsible for generating a prover and verifier key.

% Definition
\begin{definition}
   [Incrementally Verifiable Computation]\label{def:ivc}
   An 
   incrementally verifiable computation (IVC)  
   scheme is defined by
   PPT algorithms 
   $(\mathcal{G}, \mathcal{P}, \mathcal{V})$ 
   and deterministic $\En$
   denoting the generator, 
   the prover, 
   the verifier,
   and the encoder respectively,
   with the following interface
   \begin{itemize}
     \item $\mathcal{G}(1^\lambda, N) \to \pp$: 
     on input security parameter $\lambda$ and size bounds $N$, 
     samples public parameters $\pp$.
     \item $\En(\pp, F) \to (\pk, \vk)$: 
     on input public parameters $\pp$, 
     and polynomial-time function $F$,
     deterministically produces
     a prover key $\pk$ 
     and a verifier key $\vk$.
     \item $\mathcal{P}(\pk, (i, z_0, z_{i}), \omega_{i}, \pi_{i}) \to \pi_{i+1}$: 
     on input a prover key $\pk$, 
     a counter $i$, 
     an initial input $z_0$, 
     a claimed output after $i$ iterations $z_i$,
     a non-deterministic advice $\omega_i$,
     and an IVC proof $\pi_i$ attesting to $z_i$,
     produces a new proof $\pi_{i + 1}$ attesting to $z_{i + 1} = F(z_{i}, \omega_{i})$.
     \item $\mathcal{V}(\vk, (i, z_0, z_{i}), \pi_{i}) \to \{0, 1\}$: 
     on input a verifier key $\vk$,
     a counter $i$,
     an initial input $z_0$, 
     a claimed output after $i$ iterations $z_i$,
     and an IVC proof $\pi_i$ attesting to $z_i$,
     outputs $1$ if $\pi_i$ is accepting, and 
     $0$ otherwise.
   \end{itemize}
     An IVC scheme 
     $(\mathcal{G}, \En, \mathcal{P}, \mathcal{V})$
     satisfies the following requirements.
     \begin{enumerate}
     \item Perfect Completeness:    
     For any
     PPT adversary $\mathcal{A}$
     \begin{equation*}
     \Pr
     \left[
       \begin{array}{l}
         \mathcal{V}(\vk, (i + 1, z_0, z_{i + 1}), \pi_{i + 1}) = 1 
       \end{array}
       \middle\vert
       \begin{array}{l}
         \mathsf{pp} \gets \mathcal{G}(1^{\lambda}, N),\\
         (F, (i, z_0, z_i), (\omega_i, \pi_i)) \gets \mathcal{A}(\mathsf{pp}),\\
         (\pk, \vk) \gets \En(\pp, F),\\
         z_{i + 1} \gets F(z_{i}, \omega_{i}),\\
         \mathcal{V}(\vk, i, z_0, z_{i}, \pi_{i}) = 1,\\
         \pi_{i + 1} \gets \mathcal{P}(\pk, (i, z_0, z_i), (\omega_{i}, \pi_{i}))
       \end{array}
     \right] = 1
     \end{equation*}
     where $F$ is a polynomial-time computable function represented as an arithmetic circuit.
   \item Knowledge Soundness:
   Consider constant $n \in \mathbb{N}$.
   \niksdef
   % Structure
   {F} 
   % Statement
   {(z_0, z_i)}
   % Witness
   {(\omega_0, \ldots, \omega_{n - 1})}
   % Proof Format
   {\Pi}
   % Precondition
   {\mathcal{V}(\vk, (n, z_0, z), \Pi) = 1}
   % Postcondition
   {z_n = z \text{ where }\\ z_{i + 1} \gets F(z_i, \omega_i)\\
   \forall i \in \{0, \ldots, n - 1\}
   }
   Moreover, 
     $F$ is a polynomial-time computable function represented as an arithmetic circuit.
     \item Succinctness: 
     The size of an IVC proof $\pi$ is independent of the number of iterations $i$.
   \end{enumerate}
\end{definition}

\begin{definition}[Non-Interactive Argument of Knowledge]\label{def:nark}
   Consider a relation $\R$ over 
   public parameters, structure, instance, and witness tuples.
   A non-interactive argument of knowledge for $\R$ consists of PPT algorithms
   $(\Gn, \Po,\V)$ 
   and deterministic $\En$,
   denoting the generator, 
   the prover,    
   the verifier
   and the encoder respectively with the following interface.
   \begin{itemize}
       \item $\Gn(1^{\lambda}, N) \to \pp$: 
       On input security parameter $\lambda$,
       and length parameter $N$
       samples public parameters $\pp$.
       \item $\En(\pp, \Str) \to (\pk, \vk)$: 
       On input structure $\Str$, 
       representing common structure among instances,
       outputs the prover key $\pk$ and verifier key $\vk$.
       \item $\Po(\pk, u, w) \to \pi$: On input instance $u$ and
         witness $w$, outputs a proof $\pi$ proving that $(\pp, \Str, u, w) \in \R$.
       \item $\V(\vk, u, \pi) \to \{0, 1\}$: 
       Checks instance $u$ 
       given proof $\pi$.
   \end{itemize}
   An argument of knowledge satisfies \textit{completeness} if for any PPT adversary $\mathcal{A}$,
   \begin{align*}
   \Pr
   \left[
       \begin{array}{l}
       \V(\vk, u, \pi) = 1
       \end{array}
       \middle\vert
       \begin{array}{l}
       \pp \gets \Gn(1^{\lambda}, N),\\
       (\Str,(u, w)) \gets \mathcal{A}(\pp),\\
       (\pp, \Str, u, w) \in \R,\\
       (\pk, \vk) \gets \En(\pp, \Str),\\
       \pi \gets \Po(\pk, u, w)
       \end{array}
       \right]
   = 1.
   \end{align*}
   An argument of knowledge satisfies \textit{knowledge soundness} if for all PPT adversaries $\Po^*$ there exists a PPT extractor $\mathcal{E}$ such that for all randomness $\mathsf{r}$
   \[
   \Pr
   \left[
       \begin{array}{l}
       (\pp, \Str, u, w) \in \R
       \end{array}
       \middle\vert
       \begin{array}{l}
       \pp \gets \Gn(1^{\lambda}, N),\\
       (\Str,u,\pi) \gets \Po^*(\pp, \mathsf{r}),\\
       (\pk, \vk) \gets \En(\pp, \Str),\\
       \V(\vk, u, \pi) = 1,\\
       w \gets \mathcal{E}(\pp, \mathsf{r})
       \end{array}
       \right]
   \approx 1.
   \]
\end{definition}

\begin{definition}[Succinctness]
   A non-interactive argument system is succinct if the size of the proof $\pi$
   is polylogarithmic in the size of the witness $w$.
\end{definition}

\begin{definition}[Commitment Scheme]\label{def:commitment}
   A commitment scheme is defined by polynomial-time algorithm
   $\Gen : \mathbb{N}^2 \to P$
   that produces public parameters given the security parameter and size parameter, 
   a deterministic polynomial-time algorithm
   $\Commit : P \times M \times R \to C$
   that produces a commitment in $C$ given a public parameters, message, and randomness tuple
   such that 
   binding holds.
   That is, 
   for any $\mathsf{PPT}$ adversary $\mathcal{A}$,
   given
   $\pp \gets \Gen(\lambda, n)$,
   and given $((m_1, r_1), (m_2, r_2)) \gets \mathcal{A}(\pp)$
   we have that
   \[
       \Pr[(m_1, r_1) \neq (m_2, r_2) \land \Commit(\pp, m_1, r_1) = \Commit(\pp, m_2, r_2)] \approx 0.
   \]
   %
   The commitment scheme is deterministic if $\Commit$ does not use its randomness. 
 \end{definition}

 \begin{definition}[Circuit Satisfiability]
   We define the circuit satisfiability relation $\mathsf{CSAT}$
   over structure, instance, witness tuples
   as follows.
   \begin{equation*}
     \mathsf{CSAT}
     = 
     \left\{
     \begin{array}{l}
         % Statment, Witness
         (C, (x, y), w)
     \end{array}
     \middle\vert
     \begin{array}{l}
         C(x, w) = y
     \end{array}
     \right\}.
   \end{equation*}  
 \end{definition}

\subsection{Construction}

% Formal Construction

\begin{construction}[Incrementally Verifiable Computation]\label{cons:ivc}
   Given a 
   a succinct commitment scheme
   $(\Gen, \Commit)$
   and a
   succinct non-interactive argument of knowledge 
   $\snark$ for circuit-satisfiability
   we construct an IVC scheme as follows.
   
   %% Function F  
   Consider an arithmetic circuit $F$ 
   that takes non-deterministic input
   We begin by defining an augmented circuit
   $\FP$
   as follows,
   where all input arguments are taken as non-deterministic advice.
   %
   \begin{mdframed}[nobreak=true]
     \noindent \underline{$\FP(\io_i, \omega_i, \pi_i)$}:
     \begin{enumerate}
       \item Parse $(\vk_\snark, i, z_0, z_i) \gets \io_i$.
       \item  If $i = 0$:
       \begin{enumerate}
         \item Check that $z_0 = z_i$.
         \label{ivc:fp:base}
       \end{enumerate}
       \item Otherwise:
       \begin{enumerate} 
         \item Check that $\snark.\V(\vk_\snark, \io_i, \pi_i) = 1$.
         \label{ivc:fp:check:general}
       \end{enumerate}
       \item Output $\io_{i + 1} \gets (\vk_{\snark}, i + 1, z_0, F(z_i, \omega_i))$.
       \label{ivc:fp:output}
     \end{enumerate}  
   \end{mdframed} 
   %
   %
   Given the augmented circuit $\FP$,
   we define $(\Gn, \En, \Po, \V)$ as follows.
   \begin{mdframed}[nobreak=true]
     \underline{$\Gn(\lambda, N)$}:
     \begin{enumerate}
       \item Output $\pp \gets \snark.\mathcal{G}({\lambda}, N)$.
     \end{enumerate}
   \end{mdframed}
   %
   \begin{mdframed}[nobreak=true]
     \underline{$\En(\pp, F)$}:
     \begin{enumerate}
       
       \item Compute 
       $(\pk_\snark, \vk_\snark) \gets \snark.\En(\pp, \FP)$.
       \item Output $\pk \gets (\FP, \pk_\snark, \vk_\snark)$ and $\vk \gets \vk_\snark$.
     \end{enumerate}
   \end{mdframed}
   \begin{mdframed}[nobreak=true]
     \underline{$\Po(\pk, (i, z_0, z_i), (\omega_i, \pi_i))$}:
     \begin{enumerate} 
       \item Parse $(\FP, \pk_\snark, \vk_\snark) \gets \pk$.
       \item Compute
       $\io_{i + 1} \gets \FP(\vk_{\snark},
       (i, z_0, z_i), \omega_i, \pi_i)$.
       \item Let $\io_i \gets (\vk_{\snark}, i, z_0, z_i)$
       \label{ivc:prover:io}
       \item Output
       $
       \pi_{i + 1} \gets \snark.\Po(\pk_\snark, (\bot, \io_{i + 1}), (\io_i
       , \omega_i, \pi_i))
       $.
       \label{ivc:prover:proof}
     \end{enumerate} 
   \end{mdframed}
   %
   \begin{mdframed}[nobreak=true]
     \underline{$\V(\vk, (i, z_0, z_i), \pi_i)$}:
     \begin{enumerate}
       \item If $i = 0$: Check that $z_i = z_0$.
       \label{ivc:verifier:base}
       \item Otherwise:
       \begin{enumerate}
         \item Parse $\vk_{\snark} \gets \vk$.
         \item Let $\io_i \gets (\vk_\snark, i, z_0, z_i)$.
         \label{ivc:v:check:first}
         \item Check that
         $\snark.\V(\vk_\snark, (\bot, \io_i), \pi_i) = 1$.
         \label{ivc:v:check:second}
       \end{enumerate}
       
     \end{enumerate}
   \end{mdframed}
 \end{construction}

\begin{lemma}[Completeness]
 Construction~\ref{cons:ivc}
 is complete.
\end{lemma}
\newcommand{\proof}{\noindent{\bf Proof. }} %% To begin a proof write \proof

\begin{proof}
   % Adversary
 Consider arbitrary PPT adversary $\mathcal{A}$.
 % public parameters
 Suppose $\pp \gets \Gen(1^{\lambda}, N)$.
 % Structure, Instance, Witness
 Suppose that
 \[
 (F, (z_0, z_i, i), \pi_i) \gets \mathcal{A}(\pp).
 \]
 % Prover Key, Verifier Key
 Suppose that for
   $(\pk, \vk) \gets \En(\pp, F)$
 % Precondition
 we have that
 \begin{align}\label{ivc:completeness:precondition:verifier}
   \V(\vk, (z_0, z_{i}, i), \pi_{i}) = 1.
 \end{align}
 % Postcondition
 Then,
 given
 \begin{align*}
   z_{i + 1} \gets F(z_{i}, \omega_{i})
 \end{align*}
 and 
 \begin{align*}
   \pi_{i + 1} \gets \Po(\pk, (z_0, z_i, i), (\omega_i, \pi_i))
 \end{align*}
 we must show that
 \begin{align}\label{ivc:completeness:postcondition}
   \V(\vk, (z_0, z_{i+1}, N), \pi_{i + 1}) = 1
 \end{align}
 with probability $1$.
 %
 
 % Case i = 0
 Indeed, 
 consider the base case where $i = 0$.
 %
 Then, 
 by Precondition~\ref{ivc:completeness:precondition:verifier},
 by the verifier's check in the base case 
 (Step~\ref{ivc:verifier:base})
 we have that $z_0 = z_i$.
 %
 Therefore,
 $\Po$ can successfully compute $\io_{i + 1}$ 
 (Step~\ref{ivc:prover:io}),
 because the base case check of $\FP$ 
 (Step~\ref{ivc:fp:base}) passes.
 %
 Then,
 by construction of $\FP$
 (Step~\ref{ivc:fp:output})
 we have that 
 \[
   \io_{i + 1} = (\vk_{\snark}, i + 1, z_0, F(z_i, \omega_i))
 \]
 Moreover,
 by the completeness
 of $\snark$,
 we have that $\pi_{i + 1}$ generated by $\Po$ 
 (Step~\ref{ivc:prover:proof})
 is indeed satisfying.
 %
 Therefore, 
 both the checks of $\V$ in Steps~\ref{ivc:v:check:first} and~\ref{ivc:v:check:second}
 are passing.
 %
 As such,
 we have that postcondition~\ref{ivc:completeness:postcondition}
 holds.
 
 
 % Case i \geq 1
 Suppose instead that $i \geq 1$.
 %
 by Precondition~\ref{ivc:completeness:precondition:verifier},
 by the verifier's check in the general case 
 we have that
 \begin{align*}
   \snark.\V(\vk_\snark, (\bot, \io_i), \pi_i) = 1
 \end{align*}
 for $\io_i = (\vk_\snark, i, z_0, z_i)$.
 %
 Then,
 $\Po$ can successfully compute $\io_{i + 1}$ 
 (Step~\ref{ivc:prover:io}),
 as the SNARK verifier check in $\FP$
 (Step~\ref{ivc:fp:check:general})
 holds.
 %
 %
 Once again,
 by construction of $\FP$
 (Step~\ref{ivc:fp:output})
 we have that 
 \[
   \io_{i + 1} = (\vk_{\snark}, i + 1, z_0, F(z_i, \omega_i))
 \]
 Moreover,
 by the completeness
 of $\snark$,
 we have that $\pi_{i + 1}$ generated by $\Po$ 
 (Step~\ref{ivc:prover:proof})
 is indeed satisfying.
 %
 Therefore, 
 both the checks of $\V$ in Steps~\ref{ivc:v:check:first} and~\ref{ivc:v:check:second}
 are passing.
 %
 As such,
 we have that postcondition~\ref{ivc:completeness:postcondition}
 holds.
\end{proof}

\begin{lemma}[Knowledge-Soundness]
 Construction~\ref{cons:ivc}
 is knowledge-sound.
\end{lemma}

% Proof for ivc knowledge soundness

\begin{proof}
   % Premise
   Let $n$ be a global constant.
   %
   Consider a deterministic expected polynomial-time adversary $\Po^*$.
   %
   Let $\pp \gets \Gen(1^{\lambda}, N)$.
   %
   Suppose 
   on input $\pp$ and randomness $\mathsf{r}$,
   $\Po^*$
   outputs 
   polynomial-time function $F$,
   instance $(z_0, z)$,
   and IVC proof $\pi$. 
   %
   Let $(\pk, \vk) \gets \En(\pp, F)$.
   Suppose that
   % Precondition
   \begin{align*}
     \V(\vk, (z_0, z, n), \pi) = 1.
   \end{align*}
   % Postcondition
   We must construct an expected polynomial-time extractor $\mathcal{E}$
   that, 
   with input $(\pp, \mathsf{r})$,
   outputs $(\omega_0, \ldots, \omega_{n - 1})$ 
   such that by computing
   \begin{align*}
     z_{i + 1} \gets F(z_{i}, \omega_{i})
   \end{align*}
   we have that $z_n = z$ with probability 
   $1 - \negl{\lambda}$.
   
   % Overview
   We show inductively that we can construct an expected polynomial-time extractor $\mathcal{E}_i$ that outputs 
   $((z_i, \ldots, z_{n - 1}), (\omega_i, \ldots, \omega_{n - 1}), \pi_i)$ such that for all $j \in  \{i + 1, \ldots, n\}$,
   \begin{align*}
     z_j = F(z_{j - 1}, \omega_{j - 1})
   \end{align*}
   and 
   \begin{align}\label{eq:ih:2}
     \V(\vk, z_0, z_i, \pi_i) = 1
   \end{align}
   for $z_n = z$ with probability $1 - \negl{\lambda}$.
   %
   Then, 
   because when $i = 0$,
   $\V$ checks that $z_0 = z_i$ 
   the values $(\omega_0, \ldots, \omega_{n - 1})$ 
   retrieved by $\mathcal{E} = \mathcal{E}_0$
   are such that computing  
   $z_{i + 1} = F(z_i, \omega_i)$ for all $i \geq 1$ gives $z_n = z$.
   
   At a high level,
   to construct an extractor $\mathcal{E}_{i - 1}$,
   we first assume the existence of $\mathcal{E}_i$ that satisfies the inductive hypothesis. 
   We then use $\mathcal{E}_i$ to construct an adversary 
   for the underlying succinct non-interactive argument, which we denote as $\widetilde{\Po}_{i - 1}$.
   This in turn guarantees an extractor for the underlying non-interactive argument,
   which we denote as $\widetilde{\mathcal{E}}_{i - 1}$. 
   We then use $\widetilde{\mathcal{E}}_{i - 1}$ to construct $\mathcal{E}_{i - 1}$ that satisfies the inductive hypothesis.
   
   
   % Base case
   In the base case,
   let $\mathcal{E}_n(\pp, \mathsf{r})$ 
   output 
   $(\bot, \bot, \pi_n)$ 
   where $\pi_n$ is the output of 
   $\Po^*(\pp, \mathsf{r})$.
   By the premise,
   we have that $\pi_n$ is satisfying.
   As such,
   $\mathcal{E}_n$ succeeds with probability $1 - \negl{\lambda}$ in expected polynomial-time.
   
   % Inductive Step
   For $i \geq 1$, 
   suppose we can construct an expected polynomial-time extractor 
   $\mathcal{E}_i$ 
   that outputs
   $((z_i, \ldots, z_{n - 1}), (\omega_i, \ldots, \omega_{n - 1}))$, 
   and $\pi_i$ 
   that satisfies the inductive hypothesis.
   % Construct SNARK prover
   To construct an extractor $\mathcal{E}_{i - 1}$, 
   we first construct an adversary $\widetilde{\Po}_{i - 1}$
   for the underlying SNARK as follows.
   %
   %
   %
   \begin{mdframed}[nobreak=true]
     \noindent \underline{$\widetilde{\Po}_{i - 1}(\pp, \mathsf{r})$}: 
     \begin{enumerate}
       \item Let $(F, z_0) \gets \Po^*(\pp, \mathsf{r})$
       \item Let $((z_i, \ldots, z_{n - 1}), (\omega_i, \ldots, \omega_{n -
       1}), \pi_i) \gets \mathcal{E}_i(\pp, \mathsf{r})$.
       \item Let $\vk_{\snark} \gets \snark.\En(\pp, F)$.
       \item Let $\io_i \gets (\vk_\snark, (i, z_0, z_i))$.
       \item Output $(\FP, (\bot, \io_i), \pi_i)$.
     \end{enumerate}
   \end{mdframed}
   
   % SNARK prover success probability
   We now analyze the success probability of $\widetilde{\Po}_{i - 1}$. 
   By the inductive hypothesis,
   we have that 
   \[
   \V(\vk, z_0, z_i, \pi_i) = 1,
   \]
   where 
   $\pi_i \gets \mathcal{E}_i(\pp, \mathsf{r})$ 
   with probability 
   $1 - \negl{\lambda}$.
   Therefore, 
   by the the verifier's checks 
   we have that
   \[
   \snark.\V(\vk_\snark, (\bot, \io_i), \pi_i) = 1
   \]
   for $\io_i = (\vk_\snark, (i, z_0, z_i))$
   and $\vk_\snark \gets \snark.\En(\pp, \FP)$.
   Therefore,
   $\widetilde{\Po}_{i - 1}$
   succeeds in producing a satisfying proof $\pi_i$
   for structure $\FP$ and instance $\io_i$
   with probability $1 - \negl{\lambda}$.
 
   
   % Corresponding SNARK extractor
   Then,
   by the knowledge soundness of $\snark$
   there exists an 
   expected-polynomial-time
   extractor $\widetilde{\mathcal{E}}_{i - 1}$ that
   outputs 
   $(\io_{i - 1}, \omega_{i - 1}, \pi_{i - 1})$
   such that
   $\io_i = \FP(\io_{i - 1}, \omega_{i - 1}, \pi_{i - 1})$
   with probability $1 - \negl{\lambda}$.
   
   % Constructing E_{i - 1}
   Given $\widetilde{\mathcal{E}}_{i - 1}$, 
   we construct an expected polynomial time $\mathcal{E}_{i - 1}$ as follows.
 
 
   \begin{mdframed}[nobreak=true]
   \noindent \underline{$\mathcal{E}_{i - 1}(\pp, \mathsf{r})$}: 
   \begin{enumerate}
     \item Run $\widetilde{\Po}_{i - 1}(\pp, \mathsf{r})$
     to parse
     \[
     ((z_i, \ldots, z_{n - 1}), (\omega_i, \ldots, \omega_{n -
     1}), \pi_i)
     \]
     from its internal state.
     \item Compute $(\io_{i - 1}, \omega_{i - 1}, \pi_{i - 1}) \gets \widetilde{\mathcal{E}}_{i - 1}(\pp, \mathsf{r})$.
     \item Parse $(\vk_\snark, (i - 1, z_0, z_{i - 1})) \gets \io_{i - 1}$
     \item Output $((z_{i - 1}, \ldots, z_{n - 1}), (\omega_{i - 1}, \ldots, \omega_{n - 1}), \pi_{i - 1})$. 
   \end{enumerate}
 \end{mdframed}
 
  We now reason about the success probability of $\mathcal{E}_{i - 1}$.
   We first reason that the output $(z_{i - 1}, \ldots, z_{n - 1})$, and
   $(\omega_{i - 1}, \ldots, \omega_{n - 1})$ are valid.
   By the inductive hypothesis, 
   we already have that for all $j \in \{i + 1, \ldots, n\}$,
   \begin{align*}
     z_j = F(z_{j - 1}, \omega_{j - 1})
   \end{align*}
   with probability $1 - \negl{\lambda}$.
   Moreover,
   by the success probability of $\mathcal{E}_{i - 1}$,
   we have that
   \begin{align*}
     z_{i} = F(z_{i - 1}, \omega_{i - 1})
   \end{align*}
   and that
   \[
   \snark.\V(\vk_\snark, \io_{i - 1}, \pi_{i - 1}) = 1
   \]
   where $\io_{i - 1} = (\vk_\snark, (i - 1, z_0, z_i))$
   with probability $1 - \negl{\lambda}$.
   %
   Therefore we have that $\mathcal{E}_{i - 1}$
   succeeds with probability $1 - \negl{\lambda}$
   satisfying the inductive hypothesis.
 \end{proof} 